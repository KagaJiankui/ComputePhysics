\documentclass[UTF8]{ctexart}
\usepackage{geometry}
\usepackage{indentfirst}
\usepackage{hyperref}
\usepackage{harpoon}
\usepackage{amsmath}
\usepackage{amssymb}
\usepackage{graphicx}
\usepackage{float}
\usepackage{subfigure}
\usepackage{multirow}
\usepackage{array}
\usepackage{tikz}
\usetikzlibrary{arrows, shapes, positioning, calc}
\geometry{a4paper, left=1cm, right=1cm, top=2cm, bottom=2cm}
\setlength{\parindent}{1cm}
\renewcommand\contentsname{Content}
\title{泊松方程}
\author{段元兴}
\date{\today}
\begin{document}
\maketitle
\thispagestyle{empty}
\setcounter{page}{1}
\newpage
\tableofcontents
\newpage
    \section{计算结果}
        \begin{table}[H]
            \centering
            \caption{计算结果(CPU: Ryzen9 3950x, 单线程)}
            \begin{tabular}{|c|c|c|c|}
                \hline
                规模&最大误差(格点上)&$E_{L_2}$&总耗时/ms\\
                \hline
                16&3.22E-3&4.10E-3&0.086\\
                \hline
                32&8.04E-4&1.02E-3&0.1825\\
                \hline
                64&2.01E-4&2.56E-4&0.5863\\
                \hline
                128&5.02E-5&6.40E-5&2.2284\\
                \hline
                256&1.25E-5&1.60E-5&7.8982\\
                \hline
                512&3.14E-6&4.00E-6&30.9301\\
                \hline
                1024&7.84E-7&1.00E-6&121.3196\\
                \hline
            \end{tabular}
        \end{table}
        可以发现$E_{L_2}$是随着$N$以$\dfrac{1}{N^2}$下降的, 但是计算时间也以$N^2$上升.
    \section{数值求解泊松方程}
        \indent 由于矩阵是大规模稀疏的形式, 所以采用稀疏共轭梯度方法求解来保证计算速度和准确度. 迭代过程显示对于每个过程
        仅仅进行了一次迭代就能达到$10^{-15}$以上的的精度, 仔细观察后发现是因为我的初始近似解为0, 所以在第一次迭
        代时候就直接使用了$f_{ij}$作为近似解, 而这与真解仅仅相差一个常数倍数, 所以在第二次检验的时候就直接过了.
    \section{插值得到高斯点上的值}
        \indent 这里使用基函数来线性插值得到解在点$G_i, i=0,1,2,3$上的值:
        \begin{equation}
            g_i=\sum\limits_{j=0}^3\hat{u}_j\phi_j(G_i)
        \end{equation}
        其中$i, j=0,1,2,3$分别对应与点: 左下, 右下, 左上, 右上, 与题中所给的不一样(方便使用与运算得到正负号).
    \section{网格单元上的高斯积分}
        \indent 已知在$[-1,1]\times[-1,1]$区间上的高斯积分为:
        \begin{equation}
            \int_{-1}^1\int_{-1}^1(\hat{u}-u)^2dxdy=\sum\limits_{i=0}^3(g_i-u(G_i))^2
        \end{equation}
        考虑到区间的缩放得到:
        \begin{equation}
            \int_{\Omega_{ij}}(\hat{u}-u)^2dxdy=\dfrac{1}{4N^2}\sum\limits_{i=0}^3(g_i-u(G_i))^2
        \end{equation}
    \section{总$L_2$误差$E_{L_2}$}
        \indent 直接将全部小区间得到的误差求和得到:
        \begin{equation}
            E_{L_2}=\sqrt{\sum\limits_{i,j}\dfrac{1}{4N^2}\sum\limits_{k=0}^3(g_k-u(G_k))^2}
        \end{equation}
\end{document}